\subsubsection{Merge sort}

Merge sort is probably the best-known example of divide and conquer.
It sorts the elements of an array of numbers in increasing order.

It does so by first separating the array in two halves,
then sorting each individual half,
and then merging both sorted halves into the full sorted array.

To sort the halves, the function will recursively call itself
until the bits all have length one.
(A nice property of arrays of one element is that they are always sorted.)

Here is an illustration of how the algorithm splits the array
and then merges it back:
\begin{center}
    \begin{tabu}{|c|c|c|c|c|c|c|c|}
        \hline 4&7&5&1&3&8&6&2 \\ \hline
    \end{tabu}


    \begin{tabu}{|c|c|c|c|}
        \hline 4&7&5&1 \\ \hline
    \end{tabu}
    \ 
    \begin{tabu}{|c|c|c|c|}
        \hline 3&8&6&2 \\ \hline
    \end{tabu}


    \begin{tabu}{|c|c|}
        \hline 4&7 \\ \hline
    \end{tabu}
    \ 
    \begin{tabu}{|c|c|}
        \hline 5&1 \\ \hline
    \end{tabu}
    \ 
    \begin{tabu}{|c|c|}
        \hline 3&8 \\ \hline
    \end{tabu}
    \ 
    \begin{tabu}{|c|c|}
        \hline 6&2 \\ \hline
    \end{tabu}
    
    
    \begin{tabu}{|c|}
        \hline 4 \\ \hline
    \end{tabu}
    \ 
    \begin{tabu}{|c|}
        \hline 7 \\ \hline
    \end{tabu}
    \ 
    \begin{tabu}{|c|}
        \hline 5 \\ \hline
    \end{tabu}
    \ 
    \begin{tabu}{|c|}
        \hline 1 \\ \hline
    \end{tabu}
    \ 
    \begin{tabu}{|c|}
        \hline 3 \\ \hline
    \end{tabu}
    \ 
    \begin{tabu}{|c|}
        \hline 8 \\ \hline
    \end{tabu}
    \ 
    \begin{tabu}{|c|}
        \hline 6 \\ \hline
    \end{tabu}
    \ 
    \begin{tabu}{|c|}
        \hline 2 \\ \hline
    \end{tabu}
    
    
    \begin{tabu}{|c|c|}
        \hline 4&7 \\ \hline
    \end{tabu}
    \ 
    \begin{tabu}{|c|c|}
        \hline 1&5 \\ \hline
    \end{tabu}
    \ 
    \begin{tabu}{|c|c|}
        \hline 3&8 \\ \hline
    \end{tabu}
    \ 
    \begin{tabu}{|c|c|}
        \hline 2&6 \\ \hline
    \end{tabu}
    
    
    \begin{tabu}{|c|c|c|c|}
        \hline 1&4&5&7 \\ \hline
    \end{tabu}
    \ 
    \begin{tabu}{|c|c|c|c|}
        \hline 2&3&6&8 \\ \hline
    \end{tabu}
    
    
    \begin{tabu}{|c|c|c|c|c|c|c|c|}
        \hline 1&2&3&4&5&6&7&8 \\ \hline
    \end{tabu}
\end{center}

We can see that at every step the length of the parts is divided by two,
so the algorithm has $\log_2 n$ ``levels''.

For the actual merging part, the idea is to iterate through both halves
at the same time, each time selecting the smaller of the two numbers
to go into the merged array:

\begin{verbatim}
vector<int> merge(vector<int> a, vector<int> b)
{
    vector<int> merged;
    int i=0, j=0;
    while(i < a.size() || j < b.size())
    {
        if(j == b.size() || (i < a.size() && a[i] < b[j]))
        {
            merged.push_back(a[i]);
            i++;
        }
        else
        {
            merged.push_back(b[j]);
            j++;
        }
    }
    return merged;
}
\end{verbatim}

This merging algorithm can be used in other algorithms,
and is worth remembering.

Since the merging runs in linear time, and for each of the $\log_2 n$ levels
the total length to be merged is $n$, the algorithm has $O(n \log n)$
time complexity.
