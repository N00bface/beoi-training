\subsection{Game plan}

Before a contest, you should always plan in your head what you are going to do.
That way you will get less nervous, you will know what to do at all times,
and you will avoid wasting your time on one problem, thinking or debugging.

You should at least read \emph{all} problems first,
before you begin coding anything.
Take note of the ideas you have, the algorithms and bits of algorithms you find,
the data structures you will need, tricky details, etc.
\begin{itemize}
    \item Brainstorm many possible algorithms,
        then pick the stupidest that works.
    \item Do the math: find the time and space complexity and plug in the
        numbers to see if it fits the limits.
    \item Try to break the algorithm: use special or degenerate test cases.
        It might not be as efficient as you thought it was.
    \item Order the problems: shortest job first,
        \emph{in terms of your effort}
        (shortest to longest; done it before, easy, unfamiliar, hard).
        That way you can secure some easy points and it leaves you time for the
        harder problems.
\end{itemize}

When you have chosen a problem to work on, you should:
\begin{itemize}
    \item Finalize the algorithm.
    \item Create test data for tricky cases.
    \item Write the data structures, if needed.
    \item Write the input routine and test it (print what you read).
    \item Write the output routine and test it.
    \item For more intricate algorithms (e.g. IOI), write and debug the code
        \emph{one section at a time}.
    \item Get it working and verify correctness: create many test cases
        that you solve by hand.
    \item Submit. If incorrect, go back to the previous step. Don't forget to
        comment out any debugging output or asserts (do not remove it,
        chances are you'll need it again).
    \item If it exceeds the time limit, optimize progressively and keep all
        versions. If it is still too slow, cry.
\end{itemize}
