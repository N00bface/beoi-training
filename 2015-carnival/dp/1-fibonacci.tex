\subsection{Example 1: Fibonacci sequence}
We use the Fibonacci sequence as a first example to introduce Dynamic Programming. 

We want to find the $n$-th Fibonacci number.
If we just implement the definition of the Fibonacci sequence, we get:
\begin{verbatim}
int fibo(int n)
{
    if(n==0)
        return 0;
    else if(n==1)
        return 1;
    else
        return fibo(n-1) + fibo(n-2);
}
\end{verbatim}
However, since \texttt{fibo(n)} calls both \texttt{fibo(n-1)} and
\texttt{fibo(n-2)}, the running time of that function will about double
every time $n$ increases by 1.
So the complexity is $O(2^n)$, which is quite crappy.\footnote{Actually, it's more like $O(\phi^n)$, but this doesn't affect the crappiness.}
That's because we compute the same thing \emph{many} times.
To avoid that, we just have to remember the solutions to the sub-problems,
that is, the previous Fibonacci numbers, in an array.
Then we can just fill up the array progressively, which only takes $n$ steps.
\begin{verbatim}
int fibo(int n)
{
    vector<int> f;
    f.push_back(0);
    f.push_back(1);
    for(int i=2; i<=n; i++)
        f.push_back(f[i-1] + f[i-2]);
    return f[n];
}
\end{verbatim}

We actually only need the last two Fibonacci numbers in every iteration, so we can even improve the space complexity from $O(n)$ to $O(1)$.
Note that for this particular problem, even better algorithms exist. We only show this version to illustrate dynamic programming.


