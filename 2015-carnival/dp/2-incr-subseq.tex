\subsection{Example 2: Longest increasing subsequence}

However, finding which sub-problems to solve is not always as easy,
as the second example shows.

In the longest increasing subsequence problem, you have to find the longest
increasing sequence of (not necessarily consecutive) elements in an array of length $n$.
For example, in the list $[4,7,5,1,3,2,6,8]$, longest increasing subsequences
would be $[4,5,6,8]$ or $[1,2,6,8]$. (The solution is not unique.)

Trying all subsequences and finding the longest increasing one would
take $O(2^n)$ time, as every number in the array may or may not be in
the subsequence.


A first intuition might be to start from the beginning and keep the longest
increasing subsequence so far in memory, and add to it as we go.
But the problem is that it might not find the best solution.
In $[4,7,5,1,3,2,6,8]$, for example, it will add 4, then 7, but then
it will not add any element until 8, giving $[4,7,8]$, which is not optimal.

\subsubsection{Naive recursive implementation}
Another idea might be to recursively find the longest increasing subsequence which can be concatenated to a prefix of the array.
This naive (recursive) implementation is shown in Listing~\ref{code-lis1}.
\lstinputlisting[label=code-lis1,caption=Naive recursive implementation, language=C++,firstline=4, lastline=17, tabsize=2, breaklines=true, numbers=left, float]{src/lis/naive.cpp}
However, this algorithm has a very bad time complexity.

\subsubsection{Store maximal sequence length for a certain ending value}
Actually, when we want to add some element $a[i]$ of the array $a$,
instead of trying to add it to the longest increasing subsequence
between $0$ and $i-1$, we can try to find the longest
increasing subsequence that ends with an element smaller than $a[i]$.
So we could keep a table of the longest increasing subsequences
ending with all possible values.
A possible implementation is shown in Listing~\ref{code-lis2}.
\lstinputlisting[label=code-lis2,caption=Storing maximal sequence length for a certain ending value, language=C++,firstline=6, lastline=38, tabsize=2, breaklines=true, numbers=left, float]{src/lis/victor.cpp}
Beware: this assumes that the max value in array $a$ is smaller than $n$
(we could remap the values if this is not the case).
The time complexity is $O(n^2)$, much better than $O(2^n)$,
but we can still do better (and more convenient).

\subsubsection{Store smallest ending value index for a certain sequence length}
Instead of looking at the longest subsequence for some value,
let us do the opposite
and look at the smallest ending value for which there is a subsequence
of some length.
We would then have to find the biggest length such that the ending value
is smaller than the current element.

Let's name this array \texttt{smallest\_end\_for[]}.
It will contain the \emph{index} of the smallest ending value.
If you can achieve some length with an element, then you can achieve
any shorter length, so \texttt{smallest\_end\_for} is non-decreasing.
So you can just use binary search to find the longest subsequence
there is ending with an element smaller than $a[i]$. Let's say its length is
\texttt{prev\_len}.

Then you just have to update \texttt{smallest\_end\_for[prev\_len+1]}
if the solution you've just found is better than the previous one.
You don't have to update anything else, since the lengths 1 to
\texttt{prev\_len}
can already end with smaller values.
The implementation is shown in Listing~\ref{code-lis3}.

\lstinputlisting[label=code-lis3,caption=Storing smallest ending value index for a certain sequence length, language=C++,firstline=6, lastline=38, tabsize=2, breaklines=true, numbers=left, float]{src/lis/victor.cpp}

Since the outer loop is run $n$ times and the inner loop is run
at most $\log_2 n$ times, the total complexity is $O(n\log n)$.

Note: both algorithms can be adapted to output the generated
subsequence as well, not just its length.
