\subsection{Change-making problem}

In some far away land, there are $n$ types of coins, with values $c_i$.
How can you add them up to an amount of money $m$ using the least coins
possible? You can assume one of the coins is 1.

The intuitive method for that problem is to take the biggest coin
not bigger than the amount, then the biggest coin not bigger than the rest, etc.
After all, it works quite well in real life.

But that approach is not always correct.
For example, with coins $\{1,5,8,10\}$ and $m=13$, this algorithm would find
$10+1+1+1$ while the best solution is $5+8$.
It actually depends on the coin system that is used, and ours works fine
($\{1,2,5,10,\ldots\}$).

Actually, since we can find the solution from a few of the solutions
with lower $m$, we can use DP to compute the minimum number of coins
to make all amounts from 1 to $m$.
The complexity is $O(nm)$:
\begin{verbatim}
int best(vector<int> c, int m)
{
    vector<int> num_coins;
    for(int i=0; i<=m; i++)
    {
        num_coins.push_back(i);
        for(int j=0; j<c.size(); j++)
            if(i >= c[j])
                num_coins[i] = min(num_coins[i],
                                   num_coins[i-c[j]]+1);
    }
}
\end{verbatim}
Again, the algorithm can be slightly tweaked to return the actual
set of coins that is needed.
