\subsection{Longest increasing subsequence}

But finding the sub-problems to solve is not always as easy,
as the following example shows.

In the longest increasing subsequence problem, you have to find the longest
increasing sequence of elements in an array of length $n$,
not necessarily consecutive.
For example, in the list $[4,7,5,1,3,2,6,8]$, longest increasing subsequences
would be $[4,5,6,8]$ or $[1,2,6,8]$. (The solution is not unique.)

Trying all subsequences and finding the longest increasing one would
take $O(2^n)$ time, as every number in the array may or may not be in
the subsequence.

A first intuition might be to start from the beginning and keep the longest
increasing subsequence so far in memory, and add to it as we go.
But the problem is that it might not find the best solution.
In $[4,7,5,1,3,2,6,8]$, for example, it will add 4, then 7, but then
it will not add any element until 8, giving $[4,7,8]$, which is not optimal.

Actually, when we want to add some element $a[i]$ of the array $a$,
instead of trying to add it to the longest increasing subsequence
between $0$ and $i-1$, we should try to find the longest
increasing subsequence that ends with an element smaller than $a[i]$.
So we could keep a table of the longest increasing subsequences
ending with all possible values.
The implementation would look like that:
\begin{verbatim}
int lis(vector<int> a)
{
    int n = a.size()
    vector<int> ending_with;
    for(int i=0; i<n; i++)
        ending_with.push_back(0);

    for(int i=0; i<n; i++)
        for(int j=0; j < a[i]; j++)
            ending_with[a[i]] = max(ending_with[a[i]],
                                    ending_with[j]+1);
}
\end{verbatim}
Beware: this assumes that the max value in array $a$ is smaller than $n$
(we can remap the values if it is not the case).
The time complexity is $O(n^2)$, much better than $O(2^n)$,
but we can still do better (and more convenient).

Instead of looking at the longest subsequence for some value,
let us do the opposite
and look at the smallest ending value for which there is a subsequence
of some length.
We would then have to find the biggest length such that the ending value
is smaller than the current element.

Let's name this array \texttt{smallest\_end\_for[]}.
It will contain the \emph{index} of the smallest ending value.
If you can achieve some length with an element, then you can achieve
any shorter length, so \texttt{smallest\_end\_for} is non-decreasing.
So you can just use binary search to find the longest subsequence
there is ending with an element smaller than. Let's say its length is
\texttt{prev\_len}.

Then you just have to update \texttt{smallest\_end\_for[prev\_len+1]}
if the solution you've just found is better than the previous one.
You don't have to update anything else, since the lengths 1 to
\texttt{prev\_len}
can already end with smaller values.

In conclusion, here is an implementation:
\begin{verbatim}
vector<int> lis(vector<int> a)
{
    int L=0; // Longest so far
    vector<int> smallest_end_for;
    for(int i=0; i<a.size(); i++)
    {
        // Binary search for the best length before a[i]
        int lower=0, upper=L+1;
        while(lower+1 < upper)
        {
            int middle = (lower+upper)/2;
            if(a[smallest_end_for[middle]] < a[i])
                lower = middle;
            else
                upper = middle;
        }
        int prev_len = lower;
        // If the length is the best so far
        if(prev_len+1 > L)
        {
            smallest_end_for.push_back(i);
            L++;
        }
        // If this value is the new smallest for this length
        else if(a[i] < a[smallest_end_for[prev_len+1]])
        {
            smallest_end_for[prev_len+1] = i;
        }
    }
}
\end{verbatim}

Since the outer loop is run $n$ times and the inner loop is run
at most $\log_2 n$ times, the total complexity is $O(n\log n)$.

Note: both algorithms can be adapted to output the generated
subsequence as well, not just its length.
