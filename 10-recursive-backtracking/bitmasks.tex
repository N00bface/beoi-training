\frame{
    \frametitle{bitmasks}
    \framesubtitle{Passing booleans}

    \begin{itemize}[<+->]
        \item Sometimes, you need to pass around booleans to recursive calls
        \item Note: not just some boolean flags, but a boolean that indicates for example if an element has been \emph{taken}
        \item How can we best do this?
        \item Passing around a \texttt{vector<bool>}: needs copying every time
        \item Better: \texttt{bitset<N>}, a statically sized boolean collection
        \item If the size if small enough ($size <= 32$ or $size <= 64$), store it in an integer
    \end{itemize}
}

\frame{
    \frametitle{bitmasks}
    \framesubtitle{Using ints}

    \begin{itemize}[<+->]
        \item Needed operations: \texttt{<<}, \texttt{\&}, \texttt{|}
        \item set at index $i$: \texttt{bitmask |= 1 << i}
        \item test at index $i$: \texttt{bitmask \& (1 << i)}
        \item less than 32 bools: use \texttt{unsigned int}
        \item less than 64 bools: use \texttt{unsigned long long}
    \end{itemize}
}
