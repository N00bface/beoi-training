\subsection{Algorithms}
\paragraph{GCD}

(\url{http://en.wikipedia.org/wiki/Euclidean_algorithm})
\begin{lstlisting}
function gcd(a, b)
    if b = 0
       	return a
    else
	    return gcd(b, a mod b)
\end{lstlisting}

\paragraph{LCM}
The least common multiple of two positive integer numbers $a$ and $b$ is the smallest positive number that is divisible by both $a$ and $b$.
\begin{lstlisting}
function lcm(a, b)
	return a*b/gcd(a,b)
\end{lstlisting}

\paragraph{Power}
CMath contains a built-in power function (pow) that returns a floating-point number. However, in many cases we need to calculate integer powers.
In the subsection about modulo arithmetic, we'll study a slightly different algorithm.
\begin{lstlisting}
function pow(base, exp)
	if exp == 0
		return 1
	tmp = pow(base, exp/2)
	tmp *= tmp
	if base % 2 == 1
		tmp *= base
	return tmp
\end{lstlisting}
\emph{Make sure that you're not calling the C Math pow function by not choosing pow as function name.}

\paragraph{Datatypes limitations} Always be careful to prevent overflow when multiplying large numbers. In many problems, you have to return the number modulo a certain number (see subsection about primes).
