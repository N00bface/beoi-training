\subsection{Basic Geometry}
\paragraph{Same side of a line}
Consider a line through the points $A(x_a, y_a, 0)$ and $B(x_b, y_b, 0)$.
Also consider two points $C(x_c, y_c, 0)$ and $D(x_d, y_d, 0)$. We want to know if $C$ and $D$ are on the same side of the line. 
Being on the same side of a line only makes sense in 2-D, that's why the third coordinate number is zero for all four points. 
We don't just write them as points in 2-D space because we need the cross product, which has not been defined in 2-D.

$C$ and $D$ are on the same side of the line iff the angles $\hat{ABC}$ and $\hat{ABD}$ are both positive or both negative (in the range $\left(-\pi, \pi\right]$).
In that range, the sines of the angles have the same sign as the angles.
When looking at the definition of the cross product, it is clear that we can use the cross product of $(A-B)\times(C-B)$ and compare it to the cross product of $(A-B)\times(D-B)$. 
If both have the same sign, the points are on the same side of the line through $A$ and $B$. Else, they're not.

\paragraph{Area of triangle}
Consider the parallelogram defined by $O(0,0,0)$, $A(x_a, y_a, z_a)$, $B(x_b, y_b, z_b)$ and $(A+B)(x_a+x_b, y_a+y_b, z_a+z_b)$. From the definition of the cross product, we see that the area of this parallelogram is the norm of the cross product $A \times B$.
The area of the triangle defined by $O(0,0,0)$, $A(x_a, y_a, z_a)$ and $B(x_b, y_b, z_b)$ is exactly half of the area of the parallelogram, so $\frac{1}{2} (A\times B)$

\paragraph{Line segment crossing}
Given: a line segment between $A$ and $B$ and another line segment between $C$ and $D$. Determine whether the two line segments cross.
The line segments cross iff $A$ and $B$ are on different sides of the second line segment and $C$ and $D$ are on different sides of the first line segment.
Using twice the formula from the first paragraph in this section, this problem can easily be solved.


More useful geometry techniques can be found at the USACO Training Program, section 3.4.