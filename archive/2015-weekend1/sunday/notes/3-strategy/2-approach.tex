\subsection{Approaching problems}

When first reading a problem, you should always keep in mind
all the techniques and approaches you know,
and try to apply them all one by one.
It's easy to get stuck in one vision of the problem.

Here is what that checklist might look like:
\begin{itemize}
    \item Read the problem statement \emph{carefully},
        read the \emph{samples}, and make sure you \emph{understand} them.
        Don't spend an hour solving a problem that doesn't exist.
    \item Focus on the limits, not only for the main variables ($N$, $M$...)
        but also for the quantities themselves.
        If they are quite small, there might be
        an algorithm that goes through all possible values.
    \item Can you use try all possibilities to find the solution?
        If not, can you use symmetries and patterns to shrink the number
        of possibilities?
    \item Can you rephrase the problem in terms of other concepts?
        Can you reduce it to some simpler problem?
    \item Does that problem remind you of any other one you solved? Are some
        parts of the solution common?
    \item Can you cut the problem in two, solve both halves and merge the
        results? If so, divide and conquer.
    \item Can you tell if the solution is on the left or right of some point?
        If so, binary search.
    \item Does solving the problem backwards help?
    \item Can you do some precomputation before to speed up something you do
        repeatedly?
\end{itemize}
