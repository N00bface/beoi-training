\subsection{George and Number}

George is a cat, so he really likes to play.
Most of all he likes to play with his array of positive integers b.
During the game, George modifies the array by using special changes.
Let's mark George's current array as $b_1,b_2,\ldots,b_{|b|}$
(record $|b|$ denotes the current length of the array).
Then one change is a sequence of actions: 
\begin{itemize}
    \item Choose two distinct indexes $i$ and $j$
        ($1\leq i,j \leq |b|,\ i \neq j$), such that $b_i \geq b_j$.
    \item Get number $v$ = concat$(b_i,b_j)$,
        where concat$(x,y)$ is a number obtained by adding number $y$
        to the end of the decimal record of number $x$.
        For example, concat$(500,10)=50010$, concat$(2,2)=22$.
    \item Add number $v$ to the end of the array.
        The length of the array will increase by one.
    \item Remove from the array numbers with indexes $i$ and $j$.
        The length of the array will decrease by two,
        and elements of the array will become re-numbered
        from 1 to current length of the array.
\end{itemize}
George played for a long time with his array $b$ and received from array $b$
an array consisting of exactly one number $p$.
Now George wants to know: what is the maximum number of elements array $b$
could contain originally?
Help him find this number.
Note that originally the array could contain only \emph{positive} integers.

\subsubsection*{Input}
The first line of the input contains a single integer $p$
($1 \leq p < 10^{100000}$).
It is guaranteed that number $p$ doesn't contain any leading zeroes.

\subsubsection*{Output}
Print an integer --- the maximum number of elements array $b$
could contain originally.

\subsubsection*{Limits}
\begin{itemize}
    \item Time limit: 1\,sec
    \item Memory limit: 256\,MB
\end{itemize}

\subsubsection*{Examples}
\begin{verbatim}
Input:
9555
Output:
4

Input:
10000000005
Output:
2

Input:
800101
Output:
3

Input:
45
Output:
1

Input:
1000000000000001223300003342220044555
Output:
17

Input:
19992000
Output:
1

Input:
310200
Output:
2
\end{verbatim}

\subsubsection*{Notes}
Let's consider the test examples:
\begin{itemize}
    \item Originally array b can be equal to $\{5,9,5,5\}$.
        The sequence of George's changes could have been:
        $\{5,9,5,5\}\rightarrow\{5,5,95\}\rightarrow\{95,55\}
        \rightarrow\{9555\}$.
    \item Originally array $b$ could be equal to $\{1000000000,5\}$.
        Please note that the array $b$ cannot contain zeros.
    \item Originally array $b$ could be equal to $\{800,10,1\}$.
    \item Originally array $b$ could be equal to $\{45\}$.
        It cannot be equal to $\{4,5\}$,
        because George can get only array $\{45\}$ from this array
        in one operation.
\end{itemize}
Note that the numbers can be very large.

\subsubsection*{Source} \url{http://codeforces.com/problemset/problem/387/C}
